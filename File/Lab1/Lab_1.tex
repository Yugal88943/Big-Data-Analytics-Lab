
% Assignment 1: Install Apache Hadoop and Setup Single Node Cluster
%-----------------------------------------------
\refstepcounter{section}
\addlabcontentsline{Assignment \thesection: Install Apache Hadoop and Setup Single Node Cluster}
{(27-01-2026)}
{\thepage}
\section*{\centering Assignment \thesection: Install Apache Hadoop and Setup Single Node Cluster}

\noindent \textbf{Objective:} To install and configure Apache Hadoop in pseudo-distributed mode and perform basic HDFS operations such as upload, delete, replication check, and permission handling.
%-----------------------------------------------
% Task 1
% -----------------------------------------------
\subsection{Step 1: Installing Java}
\subsubsection*{Updating System}
% \vspace{-1em}
\begin{verbatim}
sudo apt update
\end{verbatim}
\begin{figure}[H]
    \centering
    \vspace{-1em}
    \includegraphics[width=1.0\textwidth]{Lab1/Images/apt_update.png}
    \caption{System Update Command}
    \label{fig:system_update}
\end{figure}
\subsubsection*{Installing Java}
\begin{verbatim}
sudo apt install openjdk-11-jdk -y
\end{verbatim}
\begin{figure}[H]
    \centering
    \vspace{-1em}
    \includegraphics[width=1.0\textwidth]{Lab1/Images/jdk_install.png}
    \caption{Java Installation}
    \label{fig:java_installation}
\end{figure} 
\subsubsection*{Java Version and Path Verification}
\begin{verbatim}
java -version
readlink -f $(which java)
\end{verbatim}
\begin{figure}[H]
    \centering
    \vspace{-1em}
    \includegraphics[width=1.0\textwidth]{Lab1/Images/jdk_version.png}
    \caption{Java Version and Path Verification}
    \label{fig:java_version_verification}
\end{figure}
%-----------------------------------------------
% Task 2
%-----------------------------------------------

\subsection{Step 2: Setup SSH}
\subsubsection*{Install SSH Server}
\begin{verbatim}
sudo apt install openssh-server -y
ssh localhost
\end{verbatim}
\begin{figure}[H]
    \centering
    \vspace{-1em}
    \includegraphics[width=1.0\textwidth]{Lab1/Images/install_openssh.png}
    \caption{Install SSH Server}
    \label{fig:install_ssh_server}
\end{figure}

\subsubsection*{Generate SSH Key Pair}
\begin{verbatim}
ssh-keygen -t rsa -P ""
cat ~/.ssh/id_rsa.pub >> ~/.ssh/authorized_keys
ssh localhost
exit
\end{verbatim}
\begin{figure}[H]
    \centering
    \vspace{-1em}
    \includegraphics[width=1.0\textwidth]{Lab1/Images/ssh_key_authorized.png}
    \caption{Generate SSH Key Pair}
    \label{fig:generate_ssh_key_pair}
\end{figure}
%-----------------------------------------------
% Task 3
%-----------------------------------------------

\subsection{Step 3: Download and Configure Hadoop}
% \vspace{-0.1em}
\subsubsection*{Create Hadoop User}
\begin{verbatim}
sudo addgroup hadoop
sudo adduser --ingroup hadoop hadoop
sudo adduser hadoop sudo 
su - hadoop
\end{verbatim}
\begin{figure}[H]
    \centering
    \vspace{-1em}
    \includegraphics[width=1.0\textwidth]{Lab1/Images/hadoop_adduser.png}
    \caption{Create Hadoop User}
    \label{fig:hadoop_adduser}
\end{figure}
\subsubsection*{Download and Extract Hadoop}
\begin{verbatim}
exit
sudo cp /home/yugal/Downloads/hadoop-3.3.0.tar.gz /home/hadoop/
sudo chown hadoop:hadoop /home/hadoop/hadoop-3.3.0.tar.gz
su - hadoop
ls
tar -xvzf hadoop-3.3.0.tar.gz
\end{verbatim}
\begin{figure}[H]
    \centering
    \vspace{-1em}
    \includegraphics[width=1.0\textwidth]{Lab1/Images/hadoop_adduser.png}
    \caption{Download and Extract Hadoop}
    \label{fig:download_extract_hadoop}
\end{figure}
\subsubsection*{Moving Hadoop}
\begin{verbatim}
mv hadoop-3.3.0 hadoop
ls
\end{verbatim}
\begin{figure}[H]
    \centering
    \vspace{-1em}
    \includegraphics[width=1.0\textwidth]{Lab1/Images/move_hadoop.png}
    \caption{Moving Hadoop}
    \label{fig:move_hadoop}
\end{figure}

\subsection{Step 4: Environment Variable Configuration}
\subsubsection*{.bashrc Modification}
\begin{verbatim}
nano ~/.bashrc
% Add the following lines at the end of the file
export JAVA_HOME=/usr/lib/jvm/java-11-openjdk-amd64
export HADOOP_HOME=/home/hadoop/hadoop
export PATH=$PATH:$HADOOP_HOME/bin:$HADOOP_HOME/sbin
\end{verbatim}
\begin{figure}[H]
    \centering
    \vspace{-1em}
    \includegraphics[width=1.0\textwidth]{Lab1/Images/env1.png}
    \includegraphics[width=1.0\textwidth]{Lab1/Images/bashrc.png}
    \caption{.bashrc Modification}
    \label{fig:bashrc_modification}
\end{figure}

%-----------------------------------------------
% Result
%-----------------------------------------------

\subsection{Step 5: Hadoop XML Configuration Files}
\subsubsection*{hadoop-env.sh Modification}
\begin{verbatim}
source ~/.bashrc
echo $HADOOP_HOME
/home/hadoop/hadoop
cd $HADOOP_HOME/etc/hadoop
ls
nano hadoop-env.sh
% Add the following lines at the path 
export JAVA_HOME=/usr/lib/jvm/java-11-openjdk-amd64
\end{verbatim}
\begin{figure}[H]
    \centering
    \vspace{-1em}
    \includegraphics[width=1.0\textwidth]{Lab1/Images/hadoop-env1.png}
    \includegraphics[width=1.0\textwidth]{Lab1/Images/hadoop-env2.png}
    \caption{hadoop-env.sh Modification}
    \label{fig:hadoop-env_modification}
\end{figure}

\subsubsection*{core-site.xml Modification}
\begin{verbatim}
nano core-site.xml
% Add the following property inside <configuration> tag
<property>
  <name>fs.defaultFS</name>
  <value>hdfs://localhost:9000</value>
</property>

\end{verbatim}
\begin{figure}[H]
    \centering
    \vspace{-1em}
    \includegraphics[width=1.0\textwidth]{Lab1/Images/nano core-site1.png}
    \includegraphics[width=1.0\textwidth]{Lab1/Images/nano core-site2.png}
    \caption{core-site.xml Modification}
    \label{fig:core-site_modification}
\end{figure}

\subsubsection*{hdfs-site.xml Modification}
\begin{verbatim}
nano hdfs-site.xml
% Add the following property inside <configuration> tag
<property>
  <name>dfs.replication</name>
  <value>1</value>
</property>

<property>
  <name>dfs.namenode.name.dir</name>
  <value>file:///home/hadoop/hdfs/namenode</value>
</property>

<property>
  <name>dfs.datanode.data.dir</name>
  <value>file:///home/hadoop/hdfs/datanode</value>
</property>
\end{verbatim}
\begin{figure}[H]
    \centering
    \vspace{-1em}
    \includegraphics[width=1.0\textwidth]{Lab1/Images/nano hdfs-site1.png}
    \includegraphics[width=1.0\textwidth]{Lab1/Images/nano hdfs-site2.png}
    \caption{hdfs-site.xml Modification}
    \label{fig:hdfs-site_modification}
\end{figure}

\subsubsection*{Creating Folder for NameNode and DataNode}
\begin{verbatim}
mkdir -p ~/hdfs/namenode
mkdir -p ~/hdfs/datanode
\end{verbatim}
\begin{figure}[H]
    \centering
    \vspace{-1em}
    \includegraphics[width=1.0\textwidth]{Lab1/Images/data_name_folder.png}
    \caption{Creating Folder for NameNode and DataNode}
    \label{fig:creating_folder_namenode_datanode}
\end{figure}

\subsubsection*{mapred-site.xml Modification}
\begin{verbatim}
nano mapred-site.xml
% Add the following property inside <configuration> tag
<property>
    <name>mapreduce.framework.name</name>
    <value>yarn</value>
</property>
\end{verbatim}
\begin{figure}[H]
    \centering
    \vspace{-1em}
    \includegraphics[width=1.0\textwidth]{Lab1/Images/mapred-site1.png}
    \includegraphics[width=1.0\textwidth]{Lab1/Images/mapred-site2.png}
    \caption{mapred-site.xml Modification}
    \label{fig:mapred-site_modification}
\end{figure}

\subsubsection*{yarn-site.xml Modification}
\begin{verbatim}
nano yarn-site.xml
% Add the following property inside <configuration> tag
<property>
  <name>yarn.nodemanager.aux-services</name>
  <value>mapreduce_shuffle</value>
</property>
\end{verbatim}
\begin{figure}[H]
    \centering
    \vspace{-1em}
    \includegraphics[width=1.0\textwidth]{Lab1/Images/yarn-site1.png}
    \includegraphics[width=1.0\textwidth]{Lab1/Images/yarn-site2.png}
    \caption{yarn-site.xml Modification}
    \label{fig:yarn-site_modification}
\end{figure}

\subsection{Step 6: Format NameNode \& Start Hadoop Services}
\subsubsection*{Format NameNode}
\begin{verbatim}
hdfs namenode -format
\end{verbatim}
\begin{figure}[H]
    \centering
    \vspace{-1em}
    \includegraphics[width=1.0\textwidth]{Lab1/Images/hdfs_namenode_format.png}
    \caption{Format NameNode}
    \label{fig:hdfs_namenode_format}
\end{figure}

\subsubsection*{Start Hadoop Services}
\begin{verbatim}
start-dfs.sh
jps
\end{verbatim}
\begin{figure}[H]
    \centering
    \vspace{-1em}
    \includegraphics[width=1.0\textwidth]{Lab1/Images/start-dfs.png}
    \includegraphics[width=1.0\textwidth]{Lab1/Images/jps.png}
    \caption{Start Hadoop Services}
    \label{fig:start-dfs}
\end{figure}

\subsubsection*{Start YARN Services}
\begin{verbatim}
start-yarn.sh
jps
\end{verbatim}
\begin{figure}[H]
    \centering
    % \vspace{-1em}
    \includegraphics[width=1.0\textwidth]{Lab1/Images/yarn.png}
    \includegraphics[width=1.0\textwidth]{Lab1/Images/YARN.png}
    \includegraphics[width=1.0\textwidth]{Lab1/Images/HDFS_UI.png}
    \caption{Start YARN Services}
    \label{fig:start-yarn}
\end{figure}

\subsection{Step 7: HDFS Operations}
% \subsubsection*{Operations}
\begin{verbatim}
% Create a test file
echo "Hello Big Data" > file.txt
% Make directory in HDFS
hdfs dfs -mkdir /data
% Upload file to HDFS
hdfs dfs -put file.txt /data
% List files
hdfs dfs -ls /data
% Check blocks & replication
hdfs fsck /data/file.txt -files -blocks
% Change permission
hdfs dfs -chmod 777 /data/file.txt
% Delete file
hdfs dfs -rm /data/file.txt
hdfs dfs -ls /data
\end{verbatim}
\begin{figure}[H]
    \centering
    \includegraphics[width=\textwidth]{Lab1/Images/hdfs_ops1.png}
    \caption{HDFS Operation 1}
\end{figure}
\begin{figure}[H]
    \centering
    \includegraphics[width=\textwidth]{Lab1/Images/hdfs_ops2.png}
    \caption{HDFS Operation 2}
\end{figure}
\begin{figure}[H]
    \centering
    \includegraphics[width=\textwidth]{Lab1/Images/hdfs_res1.png}
    \caption{HDFS Result 1}
\end{figure}
\begin{figure}[H]
    \centering
    \includegraphics[width=\textwidth]{Lab1/Images/hdfs_res2.png}
    \caption{HDFS Result 2}
\end{figure}


\subsection{Result}
Hadoop was successfully installed and configured in pseudo-distributed mode, and various HDFS operations were performed successfully.

%-----------------------------------------------
% Conclusion
%-----------------------------------------------

\subsection{Conclusion}
This experiment provided hands-on experience in setting up a Big Data environment using Hadoop and understanding the working of HDFS through practical command execution.
%-----------------------------------------------
% End of Assignment 1
%-----------------------------------------------
